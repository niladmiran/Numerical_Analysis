\documentclass{article}

% if you need to pass options to natbib, use, e.g.:
%     \PassOptionsToPackage{numbers, compress}{natbib}
% before loading neurips_2020

% ready for submission
% \usepackage{neurips_2020}

% to compile a preprint version, e.g., for submission to arXiv, add add the
% [preprint] option:
%     \usepackage[preprint]{neurips_2020}

% to compile a camera-ready version, add the [final] option, e.g.:
    \usepackage[final]{neurips_2020}

% to avoid loading the natbib package, add option nonatbib:
     % \usepackage[nonatbib]{neurips_2020}

\usepackage[utf8]{inputenc} % allow utf-8 input
\usepackage[T1]{fontenc}    % use 8-bit T1 fonts
\usepackage{hyperref}       % hyperlinks
\usepackage{url}            % simple URL typesetting
\usepackage{booktabs}       % professional-quality tables
\usepackage{amsfonts}       % blackboard math symbols
\usepackage{nicefrac}       % compact symbols for 1/2, etc.
\usepackage{microtype}      % microtypography

%% Customized package by Mao Song
\usepackage{listings}
\usepackage{graphicx}
\usepackage{amsthm}
\usepackage{amsmath}
\newcommand*\diff{\mathop{}\!\mathrm{d}}
\theoremstyle{definition}
\newtheorem{solution}{Solution}

\usepackage{xcolor}

\definecolor{codegreen}{rgb}{0,0.6,0}
\definecolor{codegray}{rgb}{0.5,0.5,0.5}
\definecolor{codepurple}{rgb}{0.58,0,0.82}
\definecolor{backcolour}{rgb}{0.95,0.95,0.92}

\lstdefinestyle{mystyle}{
    backgroundcolor=\color{backcolour},   
    commentstyle=\color{codegreen},
    keywordstyle=\color{magenta},
    numberstyle=\tiny\color{codegray},
    stringstyle=\color{codepurple},
    basicstyle=\ttfamily\footnotesize,
    breakatwhitespace=false,         
    breaklines=true,                 
    captionpos=b,                    
    keepspaces=true,                 
    numbers=left,                    
    numbersep=5pt,                  
    showspaces=false,                
    showstringspaces=false,
    showtabs=false,                  
    tabsize=2
}

\lstset{style=mystyle}


\title{Homework 4 for SI211: Numerical Analysis}

% The \author macro works with any number of authors. There are two commands
% used to separate the names and addresses of multiple authors: \And and \AND.
%
% Using \And between authors leaves it to LaTeX to determine where to break the
% lines. Using \AND forces a line break at that point. So, if LaTeX puts 3 of 4
% authors names on the first line, and the last on the second line, try using
% \AND instead of \And before the third author name.

\author{%
  {Mao Song} \\
  SIST\\
  ShanghaiTech University\\
  Student ID: 2019233134 \\
  \texttt{maosong@shanghaitech.edu.cn} \\
}

\begin{document}

\maketitle

\begin{abstract}
  This is the solution for Homework 4 of SI211: Numerical Analysis, which is taught by Boris Houska.
\end{abstract}

\section{Problem 1}
Prove that for all $x\in\mathbb{R}^n$ the inequality 
\begin{equation}
  \|x\|_\infty \leq \|x\|_2\leq \|x\|_1\leq n\|x\|_\infty
\end{equation}
holds.

\begin{solution}
  Suppose $x=[x_1,\dots,x_n]^T\in\mathbb{R}^n$, then
  \begin{align*}
  \|x\|_\infty&=\max_{i=1,\dots,n}|x_i|=\sqrt{\max_{i=1,\dots,n}x_i^2}\leq \sqrt{\sum_{i=1}^{n}x_i^2}=\|x\|_2\\
  \|x\|_2&=\sqrt{\sum_{i=1}^{n}x_i^2}\leq \sqrt{\sum_{i=1}^{n}x_i^2+2\sum_{i\neq j}|x_ix_j|}=\sum_{i=1}^{n}|x_i|=\|x\|_1\\
  \|x\|_1&=\sum_{i=1}^{n}|x_i|\leq n\max_{i=1,\dots,n}|x_i|=n\|x\|_\infty
  \end{align*}
\end{solution}

  


\section{Problem 2}
Let $H, \langle \cdot,\cdot\rangle$ be a Hilbert space with norm $\|x\|=\sqrt{\langle x,x\rangle}$. Prove that $\langle x,y\rangle=0$ if and only if we have $\|x+\alpha y\|=\|x-\alpha y\|$ for all scalars $\alpha$.



\begin{solution}
If $\langle x,y\rangle=0$, then for any $\alpha\in\mathbb{R}$,
\begin{align*}
  &\langle x,x\rangle+2\alpha\langle x,y\rangle+\langle y,y\rangle=\langle x,x\rangle-2\alpha\langle x,y\rangle+\langle y,y\rangle\\
  \Rightarrow&\|x+\alpha y\|^2=\|x-\alpha y\|^2\\
  \Rightarrow&\|x+\alpha y\|=\|x-\alpha y\|
\end{align*}
where the last equality is because $\|\cdot\|$ is nonnegative.

Now if we have $\|x+\alpha y\|=\|x-\alpha y\|$ for any $\alpha\in\mathbb{R}$. Then
\begin{align*}
&\|x+\alpha y\|=\|x-\alpha y\|\\
\Rightarrow&\|x+\alpha y\|^2=\|x-\alpha y\|^2\\
\Rightarrow&\langle x,x\rangle+2\alpha\langle x,y\rangle+\langle y,y\rangle=\langle x,x\rangle-2\alpha\langle x,y\rangle+\langle y,y\rangle\\
\Rightarrow&\alpha\langle x,y\rangle=0
\end{align*}
Since $\alpha\in\mathbb{R}$ is arbitrary, we have $\langle x,y\rangle=0$.


\end{solution}




\section{Problem 3}
Prove that the Legendre polynomials
\begin{equation}
  P_n=\frac{1}{2^nn!}\frac{\partial^n}{\partial x^n}(x^2-1)^n
\end{equation}
are orthogonal with respect to $L_2$-scalar product on the interval $[-1,1]$.

\begin{solution}
The Legendre polynomials are defined by Legendre's differential equation 
\begin{equation}
  \left[(1-x^2)P_k'(x)\right]'+k(k+1)P_k(x)=0, k\in\mathbb{N}
\end{equation}
For any $m,n\in\mathbb{N}, m\neq n$, first we let $k=n$ and multiply the equation by $P_m(x)$:
\begin{equation}
  P_m(x)\left[(1-x^2)P_n'(x)\right]'+n(n+1)P_m(x)P_n(x)=0
\end{equation}
then we let $k=m$ and multiply the equation by $P_n(x)$:
\begin{equation}
  P_n(x)\left[(1-x^2)P_m'(x)\right]'+m(m+1)P_n(x)P_m(x)=0
\end{equation}
subtracting the above equations yields:
\begin{align*}
&P_m(x)\left[(1-x^2)P_n'(x)\right]'-P_n(x)\left[(1-x^2)P_m'(x)\right]'+\left[n(n+1)-m(m+1)\right]P_m(x)P_n(x)=0\\
\Rightarrow&\left\{(1-x^2)\left[P_m(x)P_n'(x)-P_n(x)P_m'(x)\right]\right\}'+\left[n(n+1)-m(m+1)\right]P_m(x)P_n(x)=0
\end{align*}
Since
\begin{align*}
&\int_{-1}^{1}\left\{(1-x^2)\left[P_m(x)P_n'(x)-P_n(x)P_m'(x)\right]\right\}'\diff x\\
=&\left\{(1-x^2)\left[P_m(x)P_n'(x)-P_n(x)P_m'(x)\right]\right\}\Big\vert_{-1}^1\\
=&0
\end{align*}
we have
\begin{equation}
  0=\int_{-1}^{1}\left[n(n+1)-m(m+1)\right]P_m(x)P_n(x)\diff x=\int_{-1}^{1}P_m(x)P_n(x)\diff x
\end{equation}
which means $P_m(x)$ and $P_n(x)$ ($n\neq m$) are orthogonal with respect to $L_2$-scalar product on the interval $[-1,1]$.




\end{solution}

\section{Problem 4}
Solve the least-squares optimization problem
\begin{equation}
  \min_{p\in P_2}\int_{1}^2|f(x)-p(x)|^2\diff x
\end{equation}
for $f(x) =e^x$ by using Legendre polynomials. Here, $P_22$ denotes the set of polynomials of order $2$.

\begin{solution}
Since the Legendre polynomials
\begin{equation}
  P_0(x)=1,\quad P_1(x)=x, \quad P_2(x)=\frac{1}{2}(3x^2-1)
\end{equation}
form an orthogonal basis of the space of polynomials of order less than $2$  in $[-1,1]$, applying the shifting function $x\to 2x-3$, we obtain an orthogonal basis of the space of polynomials of order less than $2$  in $[1,2]$:
\begin{equation}
  \tilde{P}_0(x)=1,\quad \tilde{P}_1(x)=2x-3, \quad \tilde{P}_2(x)=6x^2-18x+13
\end{equation}
after nomarlization, we have
\begin{equation}
  L_0(x)=1,\quad L_1(x)=\sqrt{3}(2x-3), \quad L_2(x)=\sqrt{5}(6x^2-18x+13)
\end{equation}
Thus $L_1(x),L_2(x), L_3(x)$ form an orthonormal basis for the space of of polynomials of order less than $2$  in $[1,2]$. The solution is then given by 
\begin{equation}
  p(x)=\sum_{i=0}^{2}c_kL_k(x)
\end{equation}
where
\begin{align*}
c_0(x)&=\int_{1}^{2}L_0(x)f(x)\diff x=e^2-e\\
c_1(x)&=\int_{1}^{2}L_1(x)f(x)\diff x=\sqrt{3}(3e-e^2)\\
c_2(x)&=\int_{1}^{2}L_2(x)f(x)\diff x=\sqrt{5}(7e^2-19e)
\end{align*}
as a result the solution is:
\begin{align*}
p(x)&=\sum_{i=0}^{2}c_kL_k(x)=e^2-e+3(3e-e^2)(2x-3)+5(7e^2-19e)(6x^2-18x+13)\\
&=3e(70ex^2-190x^2-212ex+576x+155e-421)
\end{align*}





\end{solution}

\section{Problem 5}
Solve the least-squares optimization problem
\begin{equation}
  \min_{p\in P_1}\int_{0}^\infty|f(x)-p(x)|^2e^{-x}\diff x
\end{equation}
for $f(x) =x^2$. Here, $P_1$ denotes the set of polynomials of order $1$.


\begin{solution}
There are two kind of solutions. 
\paragraph{Mehtod 1} Since the space is simple, we assume $p(x)=ax+b$, then
\begin{align*}
I&=\int_{0}^\infty|f(x)-p(x)|^2e^{-x}\diff x\\
&=\int_{0}^\infty|x^2-(ax+b)|^2e^{-x}\diff x\\
&=b^2+(2a-4)b+2a^2-12a+24\\
&=\left(b+(a-2)\right)^2+(a-4)^2+4
\end{align*}
thus $I$ is minimized when $a=4,b=-2$. which means the best approximation polynomial is given by
\begin{equation}
  p(x)=4x-2
\end{equation}

\paragraph{Method 2} Consider the Laguerre polynomials, the Laguerre polynomials are an orthonormal basis for polynomial space in $[0,\infty)$ with respect to the following inner product
\begin{equation}
  \langle f,g\rangle=\int_{0}^{\infty}f(x)g(x)e^{-x}\diff x
\end{equation}
Thus the best approximation is given by
\begin{equation}
  p(x)=c_0L_0(x)+c_1L_1(x)
\end{equation}
where $L_0(x)=1$, $L_1(x)=-x+1$, and
\begin{align*}
c_0&=\langle f(x),L_0(x)\rangle=\int_{0}^{\infty}x^2e^{-x}\diff x=2\\
c_1&=\langle f(x),L_1(x)\rangle=\int_{0}^{\infty}x^2(-x+1)e^{-x}\diff x=-4\\
\end{align*}
Thus the solution is given by
\begin{equation}
  p(x)=2-4(-x+1)=4x-2
\end{equation}
which is consistent with the Method $1$.



\end{solution}





\end{document}
